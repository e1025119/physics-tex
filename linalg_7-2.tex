\documentclass[a4paper]{article}

\usepackage[utf8]{inputenc}
\usepackage[left=0.5in, right=0.5in, top=0.5in, bottom=1in]{geometry}
\usepackage[dvipsnames]{xcolor}
\usepackage{titling}
\usepackage{hyperref}
\usepackage{mathtools}
\usepackage{amssymb}
\usepackage{enumitem}
\setlist[enumerate]{font=\bfseries}

\newcommand{\RN}[1]{%
    \textup{\uppercase\expandafter{\romannumeral#1}}%
}
\makeatletter
    \renewcommand*\env@matrix[1][\arraystretch]{%
        \edef\arraystretch{#1}%
        \hskip -\arraycolsep
        \let\@ifnextchar\new@ifnextchar
        \array{*\c@MaxMatrixCols c}}
\makeatother

 
    \title{UE Lineare Algebra f{\"u}r PhysikerInnen \\ "Sch{\"o}ne L{\"o}sung"}

    \author{Sophia Ott \\ a12518877
    \and Beatriz Falcon \\ a12515853 
    \and Tim Grunewald \\ a12506956
    \and Benedikt Chizzola \\ a12201744
    \and Christoph Gro{\ss} \\ a01025119}

    \pretitle{\begin{center}\huge}
    \posttitle{\par\end{center}\vspace{1em}}
    \preauthor{\begin{center}\large\begin{tabular}[t]{c}}
    \postauthor{\end{tabular}\par\end{center}}
    \renewcommand\maketitlehookc{\vspace{1em}}
    \predate{\begin{center}\Large}
    \postdate{\par\end{center}}

    \begin{document}
    \setlength{\droptitle}{100pt}
    \maketitle
    \clearpage

    % ----------------
    % exercise section
    % ----------------
    \begin{description}

            % --------------------
            % exercise description
            % --------------------
        \item[Aufgabe 7.2.] Betrachte die reellen Matrizen
            \[  X=\begin{pmatrix}
                4 & 2 & 3 \\   
                2 & 1 & 2 \\   
                1 & 2 & 1 \\   
            \end{pmatrix},\quad
            X'=\begin{pmatrix}
                1 & 2 & 3 \\   
                2 & 3 & 4 \\   
                -1 & 2 & 5 \\   
            \end{pmatrix}
            \]
            aus Aufgabe 6.4. Nutze die Lösung dieser Aufgabe, um wie im Beweis von Satz 4.48 in der
            Vorlesung geordnete Basen \(B, C, B', C'\) von \(\mathbb{R}^{3}\) zu bestimmen, bez{\"u}glich 
            der \(X\) und \(X'\) in Normalform sind, d.h.
            \[  [X]_{BC}=\begin{pmatrix}
                1 & 0 & 0 \\   
                0 & 1 & 0 \\   
                0 & 0 & 1 \\   
            \end{pmatrix},\quad
            [X']_{B',C'}=\begin{pmatrix}
                1 & 0 & 0 \\   
                0 & 1 & 0 \\   
                0 & 0 & 0 \\   
            \end{pmatrix}.
            \]
    \end{description}

    \bigskip
    \bigskip

    % ----------------
    % solution outline
    % ----------------
    \underline{L{\"o}sungsweg} - Folgende Schritte f{\"u}hren uns zu den gesuchten Basen:
    \begin{enumerate}
        \item W{\"a}hle beliebige geordnete Basen \(C' \in V, B' \in W\).
        \item Sei \(A=[f]_{B'C'}\) \\
            (Annahme: \(X=[X]_{Std,Std}=M^{Std}_{Std}(F_{x})\) und \(X'=[X']_{Std,Std}=M^{Std}_{Std}(F_{x'})\)
            f{\"u}r diese Aufgabe) 
        \item Schreibe \(E'_{n},A,E''_{n}\) an.    
        \item Wende elementare Zeilenumformungen an:
            \begin{enumerate}
                \item \(A \rightarrow \tilde{A}_{1}\), wobei \(\tilde{A}_{1}\) in normierter Zeilenstufenform.
                \item \(E'_{n} \rightarrow S\)
            \end{enumerate}
        \item Wende elementare Spaltenumformungen an:
            \begin{enumerate}
                \item \(\tilde{A}_{1} \rightarrow \tilde{A}_{2}\), wobei \(\tilde{A}_{2}\) in Normalform.
                \item \(E''_{n} \rightarrow T^{-1}\)
            \end{enumerate}
        \item \(SAT^{-1}=[A]_{BC}\) bez{\"u}glich neuer Basen \(B,C\).
    \end{enumerate}
    \medskip

    Anmerkungen: 
    \begin{enumerate}[label=\(\ast\)]
        \item Die obige Prozedur kann je nach verwendeten elementaren Zeilen- bzw. Spaltenumformungen unterschiedliche Ergebnisse liefern.
            Dies ist aber durch die Existenz beliebig vieler Basen gut verkraftbar.
        \item Der zugeh{\"o}rige Beweis entstammt der begleitenden Vorlesung 
            (\href{https://moodle.univie.ac.at/mod/streamlti/view.php?id=19314913}{moodle - Video on Demand} 
            (07.11.2025, ab 01:18:00)).

            F{\"u}r eine schriftliche Form kann auch \cite{fredenhagen} herangezogen werden.
    \end{enumerate}

    % --------
    % solution
    % --------
    \noindent\begin{minipage}[0.9\textheight]{\textwidth}
        \underline{L{\"o}sung}

        \bigskip
        \medskip

        % ----------------------------
        % left minipage - solution "X"
        % ----------------------------
        \fbox{
            \begin{minipage}[t][0.95\textheight][t]{0.45\textwidth}
                \begin{flushleft}\(X\):\end{flushleft}
                    \begin{flalign*} 
                        & \begin{pmatrix} 
                            1 & 0 & 0 \\
                            0 & 1 & 0 \\
                            0 & 0 & 1
                        \end{pmatrix},
                        \begin{pmatrix} 
                            4 & 2 & 3 \\
                            2 & 1 & 2 \\
                            1 & 2 & 1
                        \end{pmatrix},
                        \begin{pmatrix} 
                            1 & 0 & 0 \\
                            0 & 1 & 0 \\
                            0 & 0 & 1
                        \end{pmatrix}\quad
                        \vrule :\RN{2}-\left(\frac{1}{2}\right)\RN{1} &
                    \end{flalign*}
                    \begin{flalign*} 
                        & \begin{pmatrix} 
                            1 & 0 & 0 \\
                            -\frac{1}{2} & 1 & 0 \\
                            0 & 0 & 1
                        \end{pmatrix},
                        \begin{pmatrix} 
                            4 & 2 & 3 \\
                            0 & 0 & \frac{1}{2} \\
                            1 & 2 & 1
                        \end{pmatrix},
                        \quad{-}\quad
                        \vrule :\RN{3}-\left(\frac{1}{4}\right)\RN{1} &
                    \end{flalign*} 
                    \begin{flalign*} 
                        & \begin{pmatrix}[1.2] 
                            1 & 0 & 0 \\
                            -\frac{1}{2} & 1 & 0 \\
                            -\frac{1}{4} & 0 & 1
                        \end{pmatrix},
                        \begin{pmatrix}[1.2] 
                            4 & 2 & 3 \\
                            0 & 0 & \frac{1}{2} \\
                            0 & \frac{3}{2} & \frac{1}{4}
                        \end{pmatrix},
                        \quad{-}\quad
                        \vrule :\RN{2} \rightleftarrows \RN{3} &
                    \end{flalign*} 
                    \begin{flalign*} 
                        & \begin{pmatrix}[1.2] 
                            1 & 0 & 0 \\
                            -\frac{1}{4} & 0 & 1 \\
                            -\frac{1}{2} & 1 & 0
                        \end{pmatrix},
                        \begin{pmatrix}[1.2] 
                            4 & 2 & 3 \\
                            0 & \frac{3}{2} & \frac{1}{4} \\
                            0 & 0 & \frac{1}{2}
                        \end{pmatrix},
                        \quad{-}\quad
                        \vrule : \begin{aligned}
                            & \left(\frac{1}{4}\right)\RN{1}\\
                            & \left(\frac{2}{3}\right)\RN{2}\\
                            & 2\RN{3}
                        \end{aligned} &
                    \end{flalign*}
                    \begin{flalign*} 
                        & \underbrace{
                            \begin{pmatrix}[1.2] 
                                \frac{1}{4} & 0 & 0 \\
                                -\frac{1}{6} & 0 & \frac{2}{3} \\
                                -1 & 2 & 0
                            \end{pmatrix}}_{\textcolor{TealBlue}{S}},
                        \underbrace{\begin{pmatrix}[1.2] 
                            \colorbox{TealBlue}{1} & \frac{1}{2} & \frac{3}{4} \\
                            0 & \colorbox{TealBlue}{1} & \frac{1}{6} \\
                            0 & 0 & \colorbox{TealBlue}{1}
                        \end{pmatrix}}_{\textcolor{TealBlue}{\tilde{X}_{1}}},
                        \quad{-}\quad &
                    \end{flalign*} 
                    \textcolor{TealBlue}{\rule{\textwidth}{0.4pt}}
                    \begin{flalign*}
                        & \quad{-}\quad,
                        \begin{pmatrix}[1.2] 
                            1 & \frac{1}{2} & \frac{3}{4} \\
                            0 & 1 & \frac{1}{6} \\
                            0 & 0 & 1
                        \end{pmatrix},
                        \begin{pmatrix} 
                            1 & 0 & 0 \\
                            0 & 1 & 0 \\
                            0 & 0 & 1
                        \end{pmatrix}\qquad 
                        \vrule :\RN{2}-\left(\frac{1}{2}\right)\RN{1} &
                    \end{flalign*} 
                    \begin{flalign*}
                        & \quad{-}\quad,
                        \begin{pmatrix}[1.2] 
                            1 & 0 & \frac{3}{4} \\
                            0 & 1 & \frac{1}{6} \\
                            0 & 0 & 1
                        \end{pmatrix},
                        \begin{pmatrix} 
                            1 & -\frac{1}{2} & 0 \\
                            0 & 1 & 0 \\
                            0 & 0 & 1
                        \end{pmatrix}\qquad
                        \vrule :\RN{3}-\left(\frac{3}{4}\right)\RN{1} &
                    \end{flalign*}
                    \begin{flalign*}
                        & \quad{-}\quad,
                        \begin{pmatrix}[1.2] 
                            1 & 0 & 0 \\
                            0 & 1 & \frac{1}{6} \\
                            0 & 0 & 1
                        \end{pmatrix},
                        \begin{pmatrix}[1.2] 
                            1 & -\frac{1}{2} & -\frac{3}{4} \\
                            0 & 1 & 0 \\
                            0 & 0 & 1
                        \end{pmatrix}\qquad
                        \vrule :\RN{3}-\left(\frac{1}{6}\right)\RN{2} &
                    \end{flalign*} 
                    \begin{flalign*}
                        & \quad{-}\quad,
                        \underbrace{\begin{pmatrix}[1.2] 
                            1 & 0 & 0 \\
                            0 & 1 & 0 \\
                            0 & 0 & 1
                        \end{pmatrix}}_{\textcolor{TealBlue}{\tilde{X}_{2}}},
                        \underbrace{\begin{pmatrix}[1.2] 
                            1 & -\frac{1}{2} & -\frac{3}{4} \\
                            0 & 1 & -\frac{1}{6} \\
                            0 & 0 & 1
                        \end{pmatrix}}_{\textcolor{TealBlue}{T^{-1}}} &
                    \end{flalign*}
                \end{minipage}%
                }
                % ------------------------------    
                % right minipage - solution "X'"    
                % ------------------------------    
                \fbox{
                    \begin{minipage}[t][0.95\textheight][t]{0.5\textwidth}
                        \begin{flushleft}\(X'\):\end{flushleft}
                            \begin{flalign*} 
                                & \begin{pmatrix} 
                                    1 & 0 & 0 \\
                                    0 & 1 & 0 \\
                                    0 & 0 & 1
                                \end{pmatrix},
                                \begin{pmatrix} 
                                    1 & 2 & 3 \\
                                    2 & 3 & 4 \\
                                    -1 & 2 & 5
                                \end{pmatrix},
                                \begin{pmatrix} 
                                    1 & 0 & 0 \\
                                    0 & 1 & 0 \\
                                    0 & 0 & 1
                                \end{pmatrix}\qquad
                                \vrule :\RN{3}+\RN{1} &
                            \end{flalign*}
                            \begin{flalign*} 
                                & \begin{pmatrix} 
                                    1 & 0 & 0 \\
                                    0 & 1 & 0 \\
                                    1 & 0 & 1
                                \end{pmatrix},
                                \begin{pmatrix} 
                                    1 & 2 & 3 \\
                                    2 & 3 & 4 \\
                                    0 & 4 & 8
                                \end{pmatrix},
                                \quad{-}\quad
                                \vrule :\RN{2}-2\RN{1} &
                            \end{flalign*} 
                            \begin{flalign*} 
                                & \begin{pmatrix}[1.2] 
                                    1 & 0 & 0 \\
                                    -2 & 1 & 0 \\
                                    1 & 0 & 1
                                \end{pmatrix},
                                \begin{pmatrix}[1.2] 
                                    1 & 2 & 3 \\
                                    0 & -1 & -2 \\
                                    0 & 4 & 8
                                \end{pmatrix},
                                \quad{-}\quad
                                \vrule :\RN{3}+4\RN{2} &
                            \end{flalign*} 
                            \begin{flalign*} 
                                & \begin{pmatrix}[1.2] 
                                    1 & 0 & 0 \\
                                    -2 & 1 & 0 \\
                                    -7 & 4 & 1
                                \end{pmatrix},
                                \begin{pmatrix}[1.2] 
                                    1 & 2 & 3 \\
                                    0 & -1 & -2 \\
                                    0 & 0 & 0
                                \end{pmatrix},
                                \quad{-}\qquad
                                \vrule :(-1)\RN{2} &
                            \end{flalign*}
                            \begin{flalign*} 
                                & \underbrace{
                                    \begin{pmatrix}[1.2] 
                                        1 & 0 & 0 \\
                                        2 & -1 & 0 \\
                                        -7 & 4 & 1
                                    \end{pmatrix}}_{\textcolor{TealBlue}{S'}},
                                \underbrace{\begin{pmatrix}[1.2] 
                                    \colorbox{TealBlue}{1} & 2 & 3 \\
                                    0 & \colorbox{TealBlue}{1} & 2 \\
                                    0 & 0 & 0
                                \end{pmatrix}}_{\textcolor{TealBlue}{\tilde{X'}_{1}}},
                                \quad{-}\quad &
                            \end{flalign*} 
                            \textcolor{TealBlue}{\rule{\textwidth}{0.4pt}}
                            \begin{flalign*}
                                & \quad{-}\quad,
                                \begin{pmatrix}[1.2] 
                                    1 & 2 & 3 \\
                                    0 & 1 & 2 \\
                                    0 & 0 & 0
                                \end{pmatrix},
                                \begin{pmatrix} 
                                    1 & 0 & 0 \\
                                    0 & 1 & 0 \\
                                    0 & 0 & 1
                                \end{pmatrix}\qquad 
                                \vrule :\RN{3}-2\RN{2} &
                            \end{flalign*} 
                            \begin{flalign*}
                                & \quad{-}\quad,
                                \begin{pmatrix}[1.2] 
                                    1 & 2 & -1 \\
                                    0 & 1 & 0 \\
                                    0 & 0 & 0
                                \end{pmatrix},
                                \begin{pmatrix} 
                                    1 & 0 & 0 \\
                                    0 & 1 & -2 \\
                                    0 & 0 & 1
                                \end{pmatrix}\qquad 
                                \vrule : \begin{aligned}
                                    & \RN{2}+2\RN{1}\\
                                    & \RN{3}+\RN{1}
                                \end{aligned} &
                            \end{flalign*}
                            \begin{flalign*}
                                & \quad{-}\quad,
                                \underbrace{\begin{pmatrix}[1.2] 
                                    1 & 0 & 0 \\
                                    0 & 1 & 0 \\
                                    0 & 0 & 0
                                \end{pmatrix}}_{\textcolor{TealBlue}{\tilde{X}^{'}_{2}}},
                                \underbrace{\begin{pmatrix}[1.2] 
                                    1 & 2 & 1 \\
                                    0 & 1 & -2 \\
                                    0 & 0 & 1
                                \end{pmatrix}}_{\textcolor{TealBlue}{T^{'^{-1}}}} &
                            \end{flalign*} 
                        \end{minipage}%
                        }
                    \end{minipage}%
                    \clearpage
    
    Damit ergeben sich f{\"u}r \(X\):
    \bigskip
                    \begin{enumerate}
                        \item \( \begin{pmatrix}[1.2] 
                                \frac{1}{4} & 0 & 0 \\
                                -\frac{1}{6} & 0 & \frac{2}{3} \\
                                -1 & 2 & 0
                        \end{pmatrix}
                        \begin{pmatrix}[1.2] 
                            4 & 2 & 3 \\
                            2 & 1 & 2 \\
                            1 & 2 & 1
                        \end{pmatrix}
                        \begin{pmatrix}[1.2] 
                            1 & -\frac{1}{2} & -\frac{3}{4} \\
                            0 & 1 & -\frac{1}{6} \\
                            0 & 0 & 1
                        \end{pmatrix} = SXT^{-1} = 
                            \begin{pmatrix}
                                1 & 0 & 0 \\   
                                0 & 1 & 0 \\   
                                0 & 0 & 1 \\   
                            \end{pmatrix} = [X]_{BC}\) 
                    \item Die geordneten Basen
                        \( B=\left(
                        \begin{pmatrix}[1.2]
                            \frac{1}{4} \\
                            -\frac{1}{6} \\
                            -1
                        \end{pmatrix},
                        \begin{pmatrix}[1.2]
                            0 \\
                            0 \\
                            2
                        \end{pmatrix},
                        \begin{pmatrix}[1.2]
                            0 \\
                            \frac{2}{3} \\
                        0 \end{pmatrix} \right)
                        \) und \( C=\left(
                        \begin{pmatrix}[1.2]
                            1 \\
                            0 \\
                            0
                        \end{pmatrix},
                        \begin{pmatrix}[1.2]
                            -\frac{1}{2} \\
                            1 \\
                            0
                        \end{pmatrix},
                        \begin{pmatrix}[1.2]
                            -\frac{2}{3} \\
                            -\frac{1}{6} \\
                        1 \end{pmatrix} \right)
                        \)
                    \end{enumerate}

    und f{\"u}r \(X'\):
    \bigskip
                    
                    \begin{enumerate}
                        \item \( \begin{pmatrix}[1.2] 
                                1 & 0 & 0 \\
                                2 & -1 & 0 \\
                                -7 & 4 & 1
                        \end{pmatrix}
                        \begin{pmatrix}[1.2] 
                            1 & 2 & 3 \\
                            2 & 3 & 4 \\
                            -1 & 2 & 5
                        \end{pmatrix}
                        \begin{pmatrix}[1.2] 
                            1 & 2 & 1 \\
                            0 & 1 & -2 \\
                            0 & 0 & 1
                        \end{pmatrix} = S'X'T^{'^{-1}} = 
                            \begin{pmatrix}
                                1 & 0 & 0 \\   
                                0 & 1 & 0 \\   
                                0 & 0 & 0 \\   
                            \end{pmatrix} = [X']_{B',C'} \) 
                    \item Die geordneten Basen
                        \( B'=\left(
                        \begin{pmatrix}[1.2]
                            1 \\
                            2 \\
                            -7
                        \end{pmatrix},
                        \begin{pmatrix}[1.2]
                            0 \\
                            -1 \\
                            4
                        \end{pmatrix},
                        \begin{pmatrix}[1.2]
                            0 \\
                            0 \\
                            1 
                        \end{pmatrix} \right)
                        \) und \( C'=\left(
                        \begin{pmatrix}[1.2]
                            1 \\
                            0 \\
                            0
                        \end{pmatrix},
                        \begin{pmatrix}[1.2]
                            2 \\
                            1 \\
                            0
                        \end{pmatrix},
                        \begin{pmatrix}[1.2]
                            1 \\
                            -2 \\
                            1
                        \end{pmatrix} \right)
                        \)
                    \end{enumerate}
    
    \bigskip
    Wie man nun sehen kann, sind \([X]_{BC}, [X']_{B',C'}\) bez{\"u}glich der gefundenen Basen in Normalform.

    \clearpage

    % --------------------
    % bibliography section
    % --------------------
    \begin{thebibliography}{1}
    \bibitem{fredenhagen} Stefan Fredenhagen, 2022, \href{https://moodle.univie.ac.at/mod/resource/view.php?id=19312584}{Skriptum - Lineare Algebra}.
    \end{thebibliography}

\end{document}
