\documentclass{article}

\usepackage[utf8]{inputenc}
\usepackage[left=0.5in, right=0.5in, top=1in, bottom=1in]{geometry}
\usepackage{mathtools}
\usepackage{amssymb}
\usepackage{titling}
\usepackage{hyperref}
\usepackage{enumitem}
\setlist[enumerate]{font=\bfseries}

\title{UE Lineare Algebra f{\"u}r PhysikerInnen \\ "Sch{\"o}ne L{\"o}sung"}

\author{Sophia Ott \\ a12518877
\and Beatriz Falcon \\ a12515853 
\and Tim Grunewald \\ a12506956
\and Benedikt Chizzola \\ a12201744
\and Christoph Gro{\ss} \\ a01025119}

\pretitle{\begin{center}\huge}
\posttitle{\par\end{center}\vspace{1em}}
\preauthor{\begin{center}\large\begin{tabular}[t]{c}}
\postauthor{\end{tabular}\par\end{center}}
\renewcommand\maketitlehookc{\vspace{1em}}
\predate{\begin{center}\Large}
\postdate{\par\end{center}}

\begin{document}
\setlength{\droptitle}{-20pt}
\maketitle
\clearpage

% exercise section
\begin{description}
    
    % exercise description
    \item[Aufgabe 7.2.] Betrachte die reellen Matrizen \\
        \[  X=\begin{pmatrix}
                4 & 2 & 3 \\   
                2 & 1 & 2 \\   
                1 & 2 & 1 \\   
            \end{pmatrix},\quad
            X'=\begin{pmatrix}
                1 & 2 & 3 \\   
                2 & 3 & 4 \\   
                -1 & 2 & 5 \\   
            \end{pmatrix}
        \]
        \par aus Aufgabe 6.4. Nutze die Lösung dieser Aufgabe, um wie im Beweis von Satz 4.48 in der
        Vorlesung geordnete Basen \(B, C, B', C'\) von \(\mathbb{R}^{3}\) zu bestimmen, bez{\"u}glich 
        der \(X\) und \(X'\) in Normalform sind, d.h.
        \[  [X]_{BC}=\begin{pmatrix}
                1 & 0 & 0 \\   
                0 & 1 & 0 \\   
                0 & 0 & 1 \\   
            \end{pmatrix},\quad
            [X']_{B',C'}=\begin{pmatrix}
                1 & 0 & 0 \\   
                0 & 1 & 0 \\   
                0 & 0 & 0 \\   
            \end{pmatrix}.
        \]
        \bigskip
        \bigskip

        % solution outline
        \underline{L{\"o}sungsweg} - Folgende Schritte f{\"u}hren uns zu den gesuchten Basen: \\
        \begin{enumerate}
            \item W{\"a}hle beliebige geordnete Basen \(C' \in V, B' \in W\).
            \item Sei \(A=[f]_{B'C'}\) \\
                (Annahme: \(X=[X]_{Std,Std}=M^{Std}_{Std}(F_{x})\) und \(X'=[X']_{Std,Std}=M^{Std}_{Std}(F_{x'})\)
                f{\"u}r diese Aufgabe) 
            \item Schreibe \(E'_{n},A,E''_{n}\) an.    
            \item Wende elementare Zeilenumformungen an:
                \begin{enumerate}
                    \item \(A \rightarrow \tilde{A}_{1}\), wobei \(\tilde{A}_{1}\) in normierter Zeilenstufenform.
                    \item \(E'_{n} \rightarrow S\)
                \end{enumerate}
            \item Wende elementare Spaltenumformungen an:
                \begin{enumerate}
                    \item \(\tilde{A}_{1} \rightarrow \tilde{A}_{2}\), wobei \(\tilde{A}_{2}\) in Normalform.
                    \item \(E''_{n} \rightarrow T^{-1}\)
                \end{enumerate}
            \item \(SAT^{-1}=[A]_{BC}\) bez{\"u}glich neuer Basen \(B,C\).
        \end{enumerate}
        \medskip

        Der zugeh{\"o}rige Beweis entstammt der begleitenden Vorlesung 
        (\href{https://moodle.univie.ac.at/mod/streamlti/view.php?id=19314913}{moodle - Video on Demand}) \\
        F{\"u}r eine schriftliche Form kann auch \cite{fredenhagen} herangezogen werden. \\
        \bigskip
        \bigskip

        % solution
        \underline{L{\"o}sung} \\
        \medskip

        \fbox{
        \begin{minipage}{0.9\textwidth}
        \vspace{1em}
        
        \( \begin{pmatrix} 
        1 & 0 & 0 \\
        0 & 1 & 0 \\
        0 & 0 & 1
        \end{pmatrix},
        \begin{pmatrix} 
        4 & 2 & 3 \\
        2 & 1 & 2 \\
        1 & 2 & 1
        \end{pmatrix},
        \begin{pmatrix} 
        1 & 0 & 0 \\
        0 & 1 & 0 \\
        0 & 0 & 1
        \end{pmatrix}
        \) 
        \vrule
        
        \vspace{1em}
        \end{minipage}}

\end{description}
\clearpage

% bibliography section
\begin{thebibliography}{1}
    \bibitem{fredenhagen} Stefan Fredenhagen, 2022, \href{https://moodle.univie.ac.at/mod/resource/view.php?id=19312584}{Skriptum - Lineare Algebra}.
\end{thebibliography}

\end{document}
